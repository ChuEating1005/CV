\documentclass[10pt, letterpaper]{article}

% Packages:
\usepackage[
    ignoreheadfoot,
    top=1 cm,
    bottom=2 cm,
    left=1.5 cm,
    right=1.5 cm,
    footskip=1.0 cm,
]{geometry}
\usepackage{titlesec}
\usepackage{tabularx}
\usepackage{array}
\usepackage[dvipsnames]{xcolor}
\definecolor{primaryColor}{RGB}{0, 79, 144}
\usepackage{enumitem}
\usepackage{fontawesome5}
\usepackage{amsmath}
\usepackage[
    pdftitle={I-Ting (Benjamin) Chu's CV},
    pdfauthor={I-Ting (Benjamin) Chu},
    pdfcreator={LaTeX with RenderCV},
    colorlinks=true,
    urlcolor=primaryColor
]{hyperref}
\usepackage[pscoord]{eso-pic}
\usepackage{calc}
\usepackage{bookmark}
\usepackage{lastpage}
\usepackage{changepage}
\usepackage{paracol}
\usepackage{ifthen}
\usepackage{needspace}
\usepackage{iftex}

\ifPDFTeX
    \input{glyphtounicode}
    \pdfgentounicode=1
    \usepackage[utf8]{inputenc}
    \usepackage{lmodern}
\fi

% Reduce overall line spacing
\linespread{0.95}

\AtBeginEnvironment{adjustwidth}{\partopsep0pt}
\pagestyle{empty}
\setcounter{secnumdepth}{0}
\setlength{\parindent}{0pt}
\setlength{\topskip}{0pt}
\setlength{\columnsep}{0cm}
\makeatletter
\let\ps@customFooterStyle\ps@plain
\patchcmd{\ps@customFooterStyle}{\thepage}{
    \color{gray}\textit{\small I-Ting (Benjamin) Chu - Page \thepage{} of \pageref*{LastPage}}
}{}{}
\makeatother
\pagestyle{customFooterStyle}

\titleformat{\section}{\needspace{3\baselineskip}\bfseries\large}{}{0pt}{}[\vspace{0.8pt}\titlerule]

\titlespacing{\section}{-1pt}{0.2 cm}{0.15 cm}

\renewcommand\labelitemi{$\circ$}
\newenvironment{highlights}{
    \begin{itemize}[
        topsep=0.05 cm,
        parsep=0.05 cm,
        partopsep=0pt,
        itemsep=0pt,
        leftmargin=0.4 cm + 10pt
    ]
}{
    \end{itemize}
}

\newenvironment{onecolentry}{
    \begin{adjustwidth}{0.2 cm + 0.00001 cm}{0.2 cm + 0.00001 cm}
}{
    \end{adjustwidth}
}

\newenvironment{twocolentry}[2][]{
    \onecolentry
    \def\secondColumn{#2}
    \setcolumnwidth{\fill, 4.5 cm}
    \begin{paracol}{2}
}{
    \switchcolumn \raggedleft \secondColumn
    \end{paracol}
    \endonecolentry
}

\newenvironment{header}{
    \setlength{\topsep}{0pt}\par\kern\topsep\centering\linespread{1.5}
}{
    \par\kern\topsep
}

\newcommand{\placelastupdatedtext}{% \placetextbox{<horizontal pos>}{<vertical pos>}{<stuff>}
  \AddToShipoutPictureFG*{% Add <stuff> to current page foreground
    \put(
        \LenToUnit{\paperwidth-2 cm-0.2 cm+0.05cm},
        \LenToUnit{\paperheight-1.0 cm}
    ){\vtop{{\null}\makebox[0pt][c]{
        \small\color{gray}\textit{Last updated in September 2024}\hspace{\widthof{Last updated in September 2024}}
    }}}%
  }%
}%

% save the original href command in a new command:
\let\hrefWithoutArrow\href

% new command for external links:
\renewcommand{\href}[2]{\hrefWithoutArrow{#1}{\ifthenelse{\equal{#2}{}}{ }{#2 }\raisebox{.15ex}{\footnotesize \faExternalLink*}}}

\hypersetup{
    colorlinks=true,
    allcolors=black,
    urlcolor=black,
    linkcolor=black,
    citecolor=black
}

\begin{document}
    \begin{header}
        \textbf{\fontsize{20 pt}{20 pt}\selectfont I-Ting (Benjamin) Chu}

        \vspace{0.3 cm}

        \normalsize
        \mbox{{\color{black}\footnotesize\faMapMarker*}\hspace*{0.13cm}Hsinchu, Taiwan}%
        \kern 0.25 cm%
        \mbox{\hrefWithoutArrow{mailto:itingchu1005@gmail.com}{\color{black}{\footnotesize\faEnvelope[regular]}\hspace*{0.13cm}itingchu1005@gmail.com}}%
        \kern 0.25 cm%
        \mbox{\hrefWithoutArrow{tel:+886-970-573-793}{\color{black}{\footnotesize\faPhone*}\hspace*{0.13cm}+886-970-573-793}}%
        \kern 0.25 cm%
        \mbox{\hrefWithoutArrow{https://linkedin.com/in/itingchu}{\color{black}{\footnotesize\faLinkedinIn}\hspace*{0.13cm}I-TING CHU}}%
        \kern 0.25 cm%
        \mbox{\hrefWithoutArrow{https://github.com/ChuEating1005}{\color{black}{\footnotesize\faGithub}\hspace*{0.13cm}ChuEating1005}}%
    \end{header}

    \vspace{0.3 cm}

    \section{Education}
        \begin{twocolentry}{
        \textit{Jul. 2022 – Present}}
            \textbf{National Yang Ming Chiao Tung University, Hsinchu, Taiwan}
        \end{twocolentry}
        \vspace{0.10 cm}
        \begin{onecolentry}
            \begin{highlights}
                \item GPA: 4.20/4.3; Ranking: 18/193 (9.33\%); (Transcript \href{https://drive.google.com/file/d/1O5Wa4WsjQL2HhvUYLbiBmsxBPsVETnsX/view?usp=sharing}{Link})
                \item Major in Computer Science; Cross-Disciplinary Specialty: Design and Innovative Technology Program
                \item Core Courses: \href{https://github.com/ChuEating1005/NYCU-Coursework}{Link}
            \end{highlights}
        \end{onecolentry}

    \section{Experience}
        \begin{itemize}[leftmargin=0.4cm]
            \item \begin{twocolentry}{\textit{Mar. 2025 - Present}}
                \textbf{Logitech} \\
                \textit{Product Quality Assurance Intern}
              \end{twocolentry}
              \begin{adjustwidth}{0.5cm}{0cm}
              Use YOLOv8 and other AI techniques to automatically validate keyboard layouts then generate a report, ensuring that each key's position and font comply with the standard.
              \end{adjustwidth}
            
            \item \begin{twocolentry}{\textit{Aug. 2024 - Present}}
                \textbf{Comp Photo Lab} \\
                \textit{Research Assistant}
              \end{twocolentry}
              \begin{adjustwidth}{0.5cm}{0cm}
                Advisor: Prof. Yu-Lun Liu. Research Topic: 3D Diffusion-based Generation \& Inpainting.
              \end{adjustwidth}
            
            \item \begin{twocolentry}{\textit{Oct. 2024 - Present}}
                \textbf{Office of Research and Development, NYCU} \\
                \textit{Part-time Student Assistant}
              \end{twocolentry}
              \begin{adjustwidth}{0.5cm}{0cm}
                Maintained backend data for the NYCU Scholars Website by retrieving faculty research information from multiple internal databases. Transformed the data into standardized XML format to support automated website updates.
              \end{adjustwidth}
        \end{itemize}

    \section{Projects}

        \begin{twocolentry}{
        \textit{\href{https://github.com/ChuEating1005/2024-LINE-FRESH}{Link}}}
            \textbf{YStory House - 2024 LINE FRESH} 
        \end{twocolentry}
        \vspace{0.05 cm}
        \begin{onecolentry}
            \begin{highlights}
                \item Developed a cross-generational social platform using Django backend and LINE Frontend Framework (LIFF)
                \item Integrated multiple APIs including ChatGPT, DALL-E, LINE Messaging API, and Speech-to-Text Model
                \item Tool Used: Python, Django, Docker, MySQL, LINE API, Firebase
            \end{highlights}
        \end{onecolentry}
        
        % \vspace{0.10 cm}
        % \begin{twocolentry}{
        % \textit{\href{https://github.com/ChuEating1005/VizGDP}{Link}}}
        %     \textbf{VizGDP - Final Project for Data Visualization} 
        % \end{twocolentry}
        % \vspace{0.10 cm}
        % \begin{onecolentry}
        %     \begin{highlights}
        %         \item Developed an interactive data visualization platform for global development indicators including GDP, population, happiness index, and life expectancy
        %         \item Implemented multiple visualization types: Bar Chart Race, Scatter Plot, Choropleth Map, and Timeline of historical economic events
        %         \item Tools Used: Vue 3, Vite, D3.js, Tailwind CSS
        %     \end{highlights}
        % \end{onecolentry}
        
        \vspace{0.10 cm}
        \begin{twocolentry}{
        \textit{\href{https://github.com/ChuEating1005/Exposure-Correction}{Link}}}
            \textbf{Exposure Correction - Final Project for Image Processing} 
        \end{twocolentry}
        \vspace{0.05 cm}
        \begin{onecolentry}
            \begin{highlights}
                \item Restore over-exposed and under-exposed images using modified Zero-DCE framework
                \item Designed a multi-scale pipeline with Laplacian pyramid decomposition and progressive enhancement techniques
                \item Tools Used: Python, PyTorch, CUDA, OpenCV
            \end{highlights}
        \end{onecolentry}
        
        \vspace{0.10 cm}
        \begin{twocolentry}{
        \textit{\href{https://github.com/ChuEating1005/Homework-Helper}{Link}}}
            \textbf{Homework Helper - Final Project for Introduction to AI}
        \end{twocolentry}
        \vspace{0.05 cm}
        \begin{onecolentry}
            \begin{highlights}
                \item Developed a line bot assistant that allows users to upload PDF files and ask questions, leveraging GPT-3.5-turbo for responses.
                \item Implemented a Retrieval-Augmented Generation (RAG) architecture to handle interactions with large PDF files.
                \item Integrated scheduling features with Google Calendar and Notion, enhancing task management capabilities.
                \item Tools Used: Python, LangChain, Pinecone, Redis, Linebot, Flask
            \end{highlights}
        \end{onecolentry}
        
        % \vspace{0.10 cm}
        % \begin{twocolentry}{
        % \textit{\href{https://github.com/ChuEating1005/Band-System}{Link}}}
        %     \textbf{Band System - Final Project for Introduction to Database System}
        % \end{twocolentry}
        % \vspace{0.10 cm}
        % \begin{onecolentry}
        %     \begin{highlights}
        %         \item Built a platform for musicians to find band members with filters and messaging features.
        %         \item Created RESTful API backend with Flask.
        %         \item Tool Used: React, Flask, PostgreSQL, AWS RDS, AWS EC2
        %     \end{highlights}
        % \end{onecolentry}
    
        % \vspace{0.10 cm}
        % \begin{twocolentry}{
        % \textit{\href{https://github.com/ChuEating1005/SDN-NFV}{Link}}}
        %     \textbf{Virtual Router}
        % \end{twocolentry}
        % \vspace{0.10 cm}
        % \begin{onecolentry}
        %     \begin{highlights}
        %         \item Developed a virtual router with Proxy ARP, Unicast DHCP, and Learning Bridge using SDN technologies.
        %         \item Integrated intra- and inter-domain traffic communication.
        %         \item Tool Used: ONOS, Mininet, Docker, Java
        %     \end{highlights}
        % \end{onecolentry}



    \section{Skills}
        \begin{onecolentry}
            \textbf{Languages:} C++, C, Python, JavaScript, SQL, Shell Script
        \end{onecolentry}
        
        \vspace{0.15 cm}
        \begin{onecolentry}
            \textbf{Framework:} React, Vue, Flask, Django, Tailwind, LangChain
        \end{onecolentry}
        
        \vspace{0.15 cm}
        \begin{onecolentry}
            \textbf{Techniques:} 3D \& 2D Computer Vision, Image Processing, Network \& System Administration, Socket Programming
            % \textbf{Techniques:} System Administration, Socket Programming, CCNA, Docker, Git, Linux, FreeBSD
        \end{onecolentry}

        \vspace{0.15 cm}
        \begin{onecolentry}
            \textbf{Tools:} PyTorch, CUDA, OpenCV, YOLO, Docker, Git, Linux, FreeBSD, AWS
            % \textbf{Techniques:} System Administration, Socket Programming, CCNA, Docker, Git, Linux, FreeBSD
        \end{onecolentry}

    \section{Awards}
        \begin{onecolentry}
            \textbf{Academic Achievements Awards (Top 5\%)}: Fall 2022 (GPA 4.27), Spring 2023 (GPA 4.3)
        \end{onecolentry}
        
        \vspace{0.15 cm}
        \begin{onecolentry}
            \textbf{Foundation Academic Course Award (Top 5\%)}: Linear Algebra
        \end{onecolentry}

    \section{Extracurricular Activities}

        \begin{twocolentry}{\textit{Jun. 2024 - Present}}
                \textbf{President} | NYCU Computer Science Student Association 2024
        \end{twocolentry}
        % \vspace{0.10 cm}
        % \begin{onecolentry}
        %     \begin{highlights}
        %         \item Leading a team of 50 people, organizing various activities, including company visits, alumni experience sharing sessions, and entertainment events.
        %         \item Providing a communication channel for CS students to voice their concerns and foster relationships across different grades.
        %     \end{highlights}
        % \end{onecolentry}
        
        % \vspace{0.15 cm}
        % \begin{twocolentry}{\textit{Jul. 2023 - May. 2024}}
        %         \textbf{Member of Information Group} | 14th NYCU Computer Science Student Association
        % \end{twocolentry}
        
        \vspace{0.15 cm}
        \begin{twocolentry}{\textit{Feb. 2024 - Nov. 2024}}
                \textbf{Co-leader of Development Group} | 2024 Meichu Hackathon
        \end{twocolentry}
        % \vspace{0.10 cm}
        % \begin{onecolentry}
        %     \begin{highlights}
        %         \item Responsible for the Meichu Hackathon signup website (\href{https://2024.meichuhackathon.org/}{Link}), a Line bot for company activities on the event day, and development group project.
        %     \end{highlights}
        % \end{onecolentry}
    

\end{document}