\documentclass[10pt, letterpaper]{article}

% Packages:
\usepackage[
    ignoreheadfoot,
    top=2 cm,
    bottom=2 cm,
    left=1.5 cm,
    right=1.5 cm,
    footskip=1.0 cm,
]{geometry}
\usepackage{titlesec}
\usepackage{tabularx}
\usepackage{array}
\usepackage[dvipsnames]{xcolor}
\definecolor{primaryColor}{RGB}{0, 79, 144}
\usepackage{enumitem}
\usepackage{fontawesome5}
\usepackage{amsmath}
\usepackage[
    pdftitle={I-Ting (Benjamin) Chu's CV},
    pdfauthor={I-Ting (Benjamin) Chu},
    pdfcreator={LaTeX with RenderCV},
    colorlinks=true,
    urlcolor=primaryColor
]{hyperref}
\usepackage[pscoord]{eso-pic}
\usepackage{calc}
\usepackage{bookmark}
\usepackage{lastpage}
\usepackage{changepage}
\usepackage{paracol}
\usepackage{ifthen}
\usepackage{needspace}
\usepackage{iftex}

\ifPDFTeX
    \input{glyphtounicode}
    \pdfgentounicode=1
    \usepackage[utf8]{inputenc}
    \usepackage{lmodern}
\fi

\AtBeginEnvironment{adjustwidth}{\partopsep0pt}
\pagestyle{empty}
\setcounter{secnumdepth}{0}
\setlength{\parindent}{0pt}
\setlength{\topskip}{0pt}
\setlength{\columnsep}{0cm}
\makeatletter
\let\ps@customFooterStyle\ps@plain
\patchcmd{\ps@customFooterStyle}{\thepage}{
    \color{gray}\textit{\small I-Ting (Benjamin) Chu - Page \thepage{} of \pageref*{LastPage}}
}{}{}
\makeatother
\pagestyle{customFooterStyle}

\titleformat{\section}{\needspace{4\baselineskip}\bfseries\large}{}{0pt}{}[\vspace{1pt}\titlerule]

\titlespacing{\section}{-1pt}{0.3 cm}{0.2 cm}

\renewcommand\labelitemi{$\circ$}
\newenvironment{highlights}{
    \begin{itemize}[
        topsep=0.10 cm,
        parsep=0.10 cm,
        partopsep=0pt,
        itemsep=0pt,
        leftmargin=0.4 cm + 10pt
    ]
}{
    \end{itemize}
}

\newenvironment{onecolentry}{
    \begin{adjustwidth}{0.2 cm + 0.00001 cm}{0.2 cm + 0.00001 cm}
}{
    \end{adjustwidth}
}

\newenvironment{twocolentry}[2][]{
    \onecolentry
    \def\secondColumn{#2}
    \setcolumnwidth{\fill, 4.5 cm}
    \begin{paracol}{2}
}{
    \switchcolumn \raggedleft \secondColumn
    \end{paracol}
    \endonecolentry
}

\newenvironment{header}{
    \setlength{\topsep}{0pt}\par\kern\topsep\centering\linespread{1.5}
}{
    \par\kern\topsep
}

\newcommand{\placelastupdatedtext}{% \placetextbox{<horizontal pos>}{<vertical pos>}{<stuff>}
  \AddToShipoutPictureFG*{% Add <stuff> to current page foreground
    \put(
        \LenToUnit{\paperwidth-2 cm-0.2 cm+0.05cm},
        \LenToUnit{\paperheight-1.0 cm}
    ){\vtop{{\null}\makebox[0pt][c]{
        \small\color{gray}\textit{Last updated in September 2024}\hspace{\widthof{Last updated in September 2024}}
    }}}%
  }%
}%

% save the original href command in a new command:
\let\hrefWithoutArrow\href

% new command for external links:
\renewcommand{\href}[2]{\hrefWithoutArrow{#1}{\ifthenelse{\equal{#2}{}}{ }{#2 }\raisebox{.15ex}{\footnotesize \faExternalLink*}}}

\begin{document}
    \placelastupdatedtext
    \begin{header}
        \textbf{\fontsize{24 pt}{24 pt}\selectfont I-Ting (Benjamin) Chu}

        \vspace{0.3 cm}

        \normalsize
        \mbox{{\color{black}\footnotesize\faMapMarker*}\hspace*{0.13cm}Hsinchu, Taiwan}%
        \kern 0.25 cm%
        \mbox{\hrefWithoutArrow{mailto:itingchu1005@gmail.com}{\color{black}{\footnotesize\faEnvelope[regular]}\hspace*{0.13cm}itingchu1005@gmail.com}}%
        \kern 0.25 cm%
        \mbox{\hrefWithoutArrow{tel:+886-970-573-793}{\color{black}{\footnotesize\faPhone*}\hspace*{0.13cm}+886-970-573-793}}%
        \kern 0.25 cm%
        \mbox{\hrefWithoutArrow{https://linkedin.com/in/i-ting-chu-7314ab2b5}{\color{black}{\footnotesize\faLinkedinIn}\hspace*{0.13cm}I-TING CHU}}%
        \kern 0.25 cm%
        \mbox{\hrefWithoutArrow{https://github.com/ChuEating1005}{\color{black}{\footnotesize\faGithub}\hspace*{0.13cm}ChuEating1005}}%
    \end{header}

    \vspace{0.3 cm}

    \section{Education}

        \begin{twocolentry}{
        \textit{Jul. 2022 – Present}}
            \textbf{National Yang Ming Chiao Tung University, Hsinchu, Taiwan}
    
            \begin{highlights}
                \item GPA: 4.22/4.3 (Overall), 4.26/4.3 (Major); Ranking: 14/192
                \item Major in Computer Science
                \item Cross-Disciplinary Specialty: Design and Innovative Technology Program
            \end{highlights}
        \end{twocolentry}

    \section{Experience}
        \begin{twocolentry}{
        \textit{Aug. 2024 - Present}}
            \textbf{Comp Photo Lab} \\
            \textit{Research Assistant}
        \end{twocolentry}
        \vspace{0.10 cm}
        \begin{onecolentry}
            \begin{highlights}
                \item Advisor: Prof. Yu-Lun Liu
                \item Research Topic: Undecided
            \end{highlights}
        \end{onecolentry}

    \section{Projects}

        \begin{twocolentry}{
        \textit{\href{https://github.com/ChuEating1005/DS-OOP}{Link}}}
            \textbf{Text-based Dungeon RPG Game}
        \end{twocolentry}
        \vspace{0.10 cm}
        \begin{onecolentry}
            \begin{highlights}
                \item Applied Object-Oriented Programming (OOP) concepts such as encapsulation, inheritance, and polymorphism to develop game mechanics.
                \item Created UML diagrams to enhance system architecture.
                \item Tool Used: C++
            \end{highlights}
        \end{onecolentry}
        
        % \vspace{0.10 cm}
        % \begin{twocolentry}{
        % \textit{\href{https://github.com/ChuEating1005/Tetris}{Link}}}
        %     \textbf{Tetris Game}
        % \end{twocolentry}
        % \vspace{0.10 cm}
        % \begin{onecolentry}
        %     \begin{highlights}
        %         \item Developed Tetris game with AI mode using genetic algorithms.
        %         \item Implemented gameplay interface with Pygame.
        %         \item Tool Used: Python, Genetic Algorithm
        %     \end{highlights}
        % \end{onecolentry}

        
        \vspace{0.10 cm}
        \begin{twocolentry}{
        \textit{\href{https://github.com/ChuEating1005/Band-System}{Link}}}
            \textbf{Band System}
        \end{twocolentry}
        \vspace{0.10 cm}
        \begin{onecolentry}
            \begin{highlights}
                \item Built a platform for musicians to find band members with filters and messaging features.
                \item Created RESTful API backend with Flask.
                \item Tool Used: React, Flask, PostgreSQL
            \end{highlights}
        \end{onecolentry}
    
        % \vspace{0.10 cm}
        % \begin{twocolentry}{
        % \textit{\href{https://github.com/ChuEating1005/SDN-NFV}{Link}}}
        %     \textbf{Virtual Router}
        % \end{twocolentry}
        % \vspace{0.10 cm}
        % \begin{onecolentry}
        %     \begin{highlights}
        %         \item Developed a virtual router with Proxy ARP, Unicast DHCP, and Learning Bridge using SDN technologies.
        %         \item Integrated intra- and inter-domain traffic communication.
        %         \item Tool Used: ONOS, Mininet, Docker, Java
        %     \end{highlights}
        % \end{onecolentry}

        \vspace{0.10 cm}
        \begin{twocolentry}{
        \textit{\href{https://github.com/ChuEating1005/Homework-Helper}{Link}}}
            \textbf{Homework Helper}
        \end{twocolentry}
        \vspace{0.10 cm}
        \begin{onecolentry}
            \begin{highlights}
                \item A line bot homework assistant, which allows the user to upload their PDF files, and ask questions about the given PDFs.
                \item Feature: Basic Q\&A, Connecting to Google Calendar or Notion to upload those schedule.
                \item Tool Used: Python, LangChain, Pinecone, Redis, Linebot, Flask
            \end{highlights}
        \end{onecolentry}


    \section{Skills}
        \begin{onecolentry}
            \textbf{Languages:} C++, C, Python, Java, JavaScript, SQL, Shell Script
        \end{onecolentry}
        
        \vspace{0.15 cm}
        \begin{onecolentry}
            \textbf{Framework:} React, Vue, Flask, Tailwind, Boostrap
        \end{onecolentry}
        
        \vspace{0.15 cm}
        \begin{onecolentry}
            \textbf{Techniques} System Administration, Socket Programming, CCNA, Docker, Git, Linux, FreeBSD
        \end{onecolentry}

    \section{Awards}
        \begin{onecolentry}
            \textbf{Academic Achievements Awards (Top 5\%)}: Fall 2022 (GPA 4.27), Spring 2023 (GPA 4.3)
        \end{onecolentry}
        
        \vspace{0.15 cm}
        \begin{onecolentry}
            \textbf{Foundation Academic Course Award (Top 5\%)}: Linear Algebra
        \end{onecolentry}

    \section{Extracurricular Activities}

        \begin{twocolentry}{\textit{Jun. 2024 - Present}}
                \textbf{President} | NYCU Computer Science Student Association 2024
        \end{twocolentry}
        % \vspace{0.10 cm}
        % \begin{onecolentry}
        %     \begin{highlights}
        %         \item Leading a team of 50 people, organizing various activities, including company visits, alumni experience sharing sessions, and entertainment events.
        %         \item Providing a communication channel for CS students to voice their concerns and foster relationships across different grades.
        %     \end{highlights}
        % \end{onecolentry}
        
        % \vspace{0.15 cm}
        % \begin{twocolentry}{\textit{Jul. 2023 - May. 2024}}
        %         \textbf{Member of Information Group} | 14th NYCU Computer Science Student Association
        % \end{twocolentry}
        
        \vspace{0.15 cm}
        \begin{twocolentry}{\textit{Feb. 2024 - Present}}
                \textbf{Co-leader of Development Group} | 2024 Meichu Hackathon
        \end{twocolentry}
        % \vspace{0.10 cm}
        % \begin{onecolentry}
        %     \begin{highlights}
        %         \item Responsible for the Meichu Hackathon signup website (\href{https://2024.meichuhackathon.org/}{Link}), a Line bot for company activities on the event day, and development group project.
        %     \end{highlights}
        % \end{onecolentry}
    

\end{document}